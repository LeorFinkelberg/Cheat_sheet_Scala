\documentclass[%
	11pt,
	a4paper,
	utf8,
	%twocolumn
		]{article}	

\usepackage{style_packages/podvoyskiy_article_extended}


\begin{document}
\title{Заметки. Практика использования и наиболее полезные конструкции языка \texttt{Scala}}

\author{\itshape Подвойский А.О.}

\date{}
\maketitle

\thispagestyle{fancy}

Здесь приводятся заметки по некоторым вопросам, касающимся машинного обучения, анализа данных, программирования на языках \texttt{Scala} и прочим сопряженным вопросам так или иначе, затрагивающим работу с данными.


%\shorttableofcontents{Краткое содержание}{1}

\tableofcontents

\section{Установка SDK}

SDKMAN (Software Development Kit Manager) \url{https://sdkman.io/} -- Очень полезная утилита для работы scala-средой.

\begin{lstlisting}[
style = bash,
numbers = none	
]
curl -s "https://get.sdkman.io" | bash
source "$HOME/.sdkman/bin/sdkman-init.sh"
sdk version
\end{lstlisting}

\section{Установка библиотек для Scala}

Например, библиотеку \texttt{breeze} \url{https://github.com/scalanlp/breeze/wiki/Installation}, близкую по своему функционалу к библиотеке \texttt{numpy} языка \texttt{Python}, можно установить с помощью \texttt{sbt} следующим образом
\begin{lstlisting}[
style = scala,
title = {\sffamily build.sbt},
numbers = none	
]
name := "project_name"
version := "0.1"
scalaVersion := "2.13.3"

libraryDependencies ++= Seq(
    // Last stable release
    "org.scalanlp" %% "breeze" % "1.1",
    // Native libraries are not included by default. add this if you want them
    // Native libraries greatly improve performance, but increase jar sizes. 
    // It also packages various blas implementations, which have licenses that may or may not
    // be compatible with the Apache License. No GPL code, as best I know.
    "org.scalanlp" %% "breeze-natives" % "1.1",
    // The visualization library is distributed separately as well.
    // It depends on LGPL code
    "org.scalanlp" %% "breeze-viz" % "1.1"
)
\end{lstlisting}

Если используется IDE IntelliJ IDEA, то файл \texttt{build.sbt} должен располагаться в директории проекта, например, \directory{C: > Users > ADM > IdeaProjects > project\_name}. Тогда после запуска сессии IDEA будут доступны все библиотеки.

Теперь можно запустить сессию в директории с файлом \texttt{build.sbt} командной
\begin{lstlisting}[
style = bash,
numbers = none
]
$ sbt console
scala> import breeze.linalg._
scala> val v = DenseVector(1.0, 2.0, 3.0)
\end{lstlisting}

К слову, есть полезная шпаргалка по \texttt{breeze} \url{https://github.com/scalanlp/breeze/wiki/Linear-Algebra-Cheat-Sheet}

\section{Компиляция программ на Scala}

Пусть есть программа такая программа
\begin{lstlisting}[
style = scala,
numbers = none	
]
object Hello extends App {
	println("Hello, world")
}
\end{lstlisting}

Скомпилировать эту программу можно с помощью утилиты командной строки \texttt{scalac}
\begin{lstlisting}[
style = bash,
numbers = none	
]
scalac Hello.scala
\end{lstlisting}

Затем можно запустить программу с помощью утилиты командной строки \texttt{scala}
\begin{lstlisting}[
style = bash,
numbers = none	
]
scala Hello
\end{lstlisting}

После этого в рабочей директории появятся файлы с расширениями \texttt{Hello.class}, \texttt{'Hello\$.class'}, \texttt{'Hello\$delayedInit\$body.class'}

\section{Приемы использования библиотеки Breeze}

Быстрое введение в библиотеку \texttt{Breeze} можно найти здесь \url{https://github.com/scalanlp/breeze/wiki/Quickstart}.




%\listoffigures\addcontentsline{toc}{section}{Список иллюстраций}

% Источники в "Газовой промышленности" нумеруются по мере упоминания 
\begin{thebibliography}{99}\addcontentsline{toc}{section}{Список литературы}
	\bibitem{hostmann:scala-2013}{{\emph{Хостаманн К.} Scala для нетерпеливых. -- М.: ДМК Пресс, 2013. -- 408~с. }
\end{thebibliography}

\end{document}
